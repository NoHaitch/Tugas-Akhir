% ------------------------------------------------------------------- %
%  LITERATURE REVIEW CHAPTER
%  AUTHOR: Raden Francisco Trianto Bratadiningrat (@NoHaitch)
%  DATE: 2025-10-28
% ------------------------------------------------------------------- %

% ----- JUDUL BAB -----
\chapter{Kajian Pustaka}

Bab Kajian Pustaka digunakan untuk mendeskripsikan kajian literatur yang terkait dengan persoalan tugas akhir. Tujuan kajian pustaka adalah:

\begin{enumerate}
    \item Memberikan pemahaman yang cukup kepada pembaca tentang teori atau pekerjaan yang terkait langsung dengan penyelesaian persoalan
    \item Menyampaikan informasi apa saja yang sudah ditulis/dilaporkan oleh pihak lain (peneliti/Tugas Akhir/Tesis) tentang hasil penelitian/pekerjaan mereka yang sama/mirip/erat kaitannya dengan persoalan tugas akhir
    \item Menunjukkan kepada pembaca adanya gap seperti pada rumusan masalah yang memang belum terselesaikan
\end{enumerate}

\section{Contoh Subbab}

Contoh subbab.

\subsection{Contoh Referensi Sitasi}

Referensi diletakan pada format biblitext dalam file `references.bib`

Referensi disitasi menggunakan format APA dengan `cite\{\}`

Publikasi ilmiah sudah berkembang dari abad ke-17, dengan terbitnya \textit{Philosophical Transactions} oleh Royal Society. Sistem publikasi ilmiah terus berevolusi untuk menjawab kebutuhan validasi, standarisasi, dan kolaborasi antar ilmuwan dalam mendokumentasikan serta menyebarluaskan pengetahuan secara formal (\cite{fyfe2022}).

\subsubsection{Contoh Gambar}

Contoh gambar adalah sebagaimana terlihat pada Gambar~\ref{fig:contoh-inline}, 

\begin{center}
    \includegraphics[width=10cm]{images/example.jpg}
    \captionof{figure}{Tahapan konstruksi koleksi retorik kalimat}
    \label{fig:contoh-inline}
\end{center}


\subsubsubsection{Contoh Tabel}

Data dapat dilihat pada Tabel~\ref{tab:contoh-merge} berikut.

\begin{table}[!htbp]
    \centering
    \DefaultTableFormatting                
    \caption{Contoh Tabel}
    \begin{tabularx}{\textwidth}{|c|X|c|}
        \hline
        \rowcolor[HTML]{EFEFEF}
        \textbf{No.} & \textbf{Kegiatan} & \textbf{Waktu Pelaksanaan} \\
        \hline

        % Multi-row example
        \multirow{2}{*}{1} & Studi Literatur & Januari 2025 \\
        \cline{2-3}
        & Analisis Kebutuhan & Februari 2025 \\
        \hline

        % Normal row
        2 & Perancangan Sistem & Maret 2025 \\
        \hline

        % Multi-column example
        3 & \multicolumn{2}{c|}{Implementasi dan Pengujian (April–Mei 2025)} \\
        \hline

        % Normal row
        4 & Dokumentasi & Juni 2025 \\
        \hline
    \end{tabularx}
    \label{tab:contoh-merge}
\end{table}
