% ------------------------------------------------------------------- %
%  LITERATURE REVIEW CHAPTER
%  AUTHOR: Raden Francisco Trianto Bratadiningrat (@NoHaitch)
%  DATE: 2025-10-28
% ------------------------------------------------------------------- %

% ----- JUDUL BAB -----
\chapter{Kajian Pustaka}

\section{Publikasi Ilmiah}

Publikasi ilmiah mulai berkembang dari abad ke-17, dengan terbitnya \textit{Philosophical Transactions} oleh Royal Society. 
Sistem publikasi ilmiah terus berevolusi untuk menjawab kebutuhan validasi, standarisasi, dan kolaborasi antar ilmuwan dalam 
mendokumentasikan serta menyebarluaskan pengetahuan secara formal (\cite{fyfe2022}).

\section{\textit{Open Peer Review (OPR)}}

Menurut Ross-Hellauer, \textit{Open Peer Roweview} atau OPR didefinisikan sebagai istilah umum 
yang mencakup berbagai bentuk adaptasi model \textit{peer review} agar selaras dengan prinsip dan tujuan 
Sains Terbuka (\textit{Open Science}) (\cite{ross2017openpeerreview}). Ross-Hellauer mendefinisikan sifat-sifat 
OPR pada Tabel~\ref{tab:peer-review-traits}.

\begin{xltabular}{\textwidth}{|c|l|X|}
\caption{Sifat-Sifat \textit{Open Peer Review}}\label{tab:peer-review-traits} \\

\hline
\rowcolor[HTML]{EFEFEF}
\textbf{No.} & \textbf{Sifat \textit{Open Peer Review}} & \textbf{Penjelasan} \\
\hline
\endfirsthead

\multicolumn{3}{c}{{\tablename\ \thetable{} Sifat-Sifat \textit{Open Peer Review}}} \\
\hline
\rowcolor[HTML]{EFEFEF}
\textbf{No.} & \textbf{Sifat \textit{Open Peer Review}} & \textbf{Penjelasan} \\
\hline
\endhead

\hline
\multicolumn{3}{r}{{Bersambung ke halaman berikutnya}} \\
\endfoot

\hline
\endlastfoot  

1 & \textit{\textbf{Open identities}} & Penulis dan Reviewer saling mengetahui identitas satu sama lain. \\
\hline
2 & \textit{\textbf{Open reports}} & Laporan telaah dipublikasikan bersama artikel terkait yang ditelaah. \\
\hline
3 & \textit{\textbf{Open participation}} & Komunitas ilmiah umum dapat berkontribusi dalam proses penelaahan. \\
\hline
4 & \textit{\textbf{Open interaction}} & Diskusi timbal balik antara para penulis dan penelaah, didorong dan diperbolehkan. \\
\hline
5 & \textit{\textbf{Open pre-review manuscripts}} & Naskah harus sudah tersedia sebelum dimulainya prosedur formal \textit{Peer Review}. \\
\hline
6 & \textit{\textbf{Open final-version commenting}} & Versi publikasi final dapat ditelaah atau dikomentari oleh publik. \\
\hline
7 & \textit{\textbf{Open platforms (“decoupled review”)}} & Proses penelaahan difasilitasi oleh entitas organisasi yang berbeda dari entitas publikasi. \\
\end{xltabular}
