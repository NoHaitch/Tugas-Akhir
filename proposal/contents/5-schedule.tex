% ------------------------------------------------------------------- %
%  SCHEDULE CHAPTER
%  AUTHOR: Raden Francisco Trianto Bratadiningrat (@NoHaitch)
%  DATE: 2025-10-28
% ------------------------------------------------------------------- %

% ----- JUDUL BAB -----
\chapter{Rencana Pelaksanaan}

Bab Rencana Pelaksanaan digunakan untuk mendeskripsikan rencana pelaksanaan berupa jadwal dan risiko-risiko yang mungkin dihadapi dan rencana mitigasinya. Tujuan bab ini  adalah:
\begin{enumerate}
    \item Mahasiswa memiliki rencana yang jelas mengenai pelaksanaan TA
    \item Mahasiswa mengenali risiko-risiko yang mungkin dihadapi dan sudah menyiapkan diri untuk mengantisipasi risiko tersebut.
\end{enumerate}

% ----- JADWAL -----
\section{Jadwal}

Cantumkan jadwal pengerjaan tugas akhir lengkap dengan uraiannya.

% ----- RESIKO -----
\section{Risiko}

Cantumkan 5 risiko tertinggi yang mungkin dihadapi dalam pengerjaan tugas akhir. Risiko yang dicantumkan dapat merupakan risiko dari sisi teknis, risiko dari sisi operasional, risiko dari metode yang dipilih, dan sebagainya. Cantumkan pula rencana mitigasi dari risiko-risiko tersebut.